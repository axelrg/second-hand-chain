\capitulo{7}{Conclusiones y Líneas de trabajo futuras}

Durante este proyecto he podido emplear una gran cantidad de herramientas y tecnologías, por ello, en este último capítulo vamos a repasar todas estas conclusiones y además a estudiar diferentes opciones para mejorar el producto en un futuro.

\section{Conclusiones}

Se ha conseguido cumplir con los objetivos iniciales establecidos al inicio de este proyecto y además se han añadido diversas mejoras a lo inicialmente establecido:
\begin{itemize}
    \item Se ha conseguido desplegar el proyecto tanto a nivel de \textit{frontend} cómo de \textit{backend}, permitiendo a los usuarios poder interactuar con la aplicación de la forma lo más sencilla posible.
    \item Creación de unos contratos digitales que satisfacen las necesidades establecidas inicialmente, todo ello respetando unos principios de diseño que garantizan la inmutabilidad de los datos y el libre acceso a ellos.
    \item He creado un  \textit{frontend} que permite satisfacer todas las necesidades propuestas y que es capaz de interactuar con \textit{Blockchain}.
    \item Se ha implementado un sistema que permite a los usuarios añadir imágenes a los teléfonos, todo ello en coherencia con las grandes ventajas de \textit{Blockchain} y resolviendo el gran problema del coste 
    de almacenaje de gran cantidad de datos en estos sistemas.
    \item Mediante el uso de la metodología de gestión de proyectos \textit{Scrum} hemos podido gestionar el proyecto de forma que este se haya podido adaptar a la introducción de mejoras y cambios a los requisitos establecidos inicialmente.
    \item Todo el proyecto se ha realizado empleando herramientas que pueden ser consideradas \textit{state of the art}, de esta forma se facilita la introducción de mejoras en un futuro y su mantenibilidad .
    \item He adquirido una gran cantidad de conocimientos tanto a  nivel teórico  como técnico de \textit{Blockchain}, los cuales desconocía por completo antes del inicio de este proyecto.
    \item He ampliado mis conocimientos sobre programación web, principalmente en lo que respecta a \textit{React} y el resto de librerías para facilitar el estilado de la web.
    \item A nivel personal, aunque no contaba como uno de los requisitos iniciales de la aplicación, he conseguido desplegar la aplicación y hacerla accesible a cualquiera, lo cual permite poder enseñar fácilmente tu trabajo.
    \item Finalmente he descubierto que la tecnología \textit{Blockchain} tiene un gran potencial detrás y un funcionamiento muy ingenioso el cual he disfrutado aprendiendo, al poder aplicar conocimientos estudiados a lo largo del grado.
    \item No obstante, también he conocido de primera mano sus limitaciones, por las cuales es mucho más complicado que en la \textit{web2} construir aplicaciones y por eso en el estado actual de esta tecnología considero que debe tener un papel auxiliar.
    \item Por otro lado considero que las ventajas de esta tecnología son complicadas de explicar a la población general sin conocimientos de este tema. Principalmente el por que deben pagar por usar esta tecnología mientras que en una mayoría de servicios de \textit{web2} no.
\end{itemize}

\section{Líneas de trabajo futuras}

Dado que al final este proyecto tiene una fecha de entrega que hay que cumplir, como al final cualquier otro, se han quedado bastantes ideas sin poder ser implementadas, las cuales expongo a continuación:
\begin{itemize}
    \item Implementar filtrado de teléfonos, actualmente no se pueden filtrar teléfonos, esta característica sería sin duda la primera que implementaría, debido a su impacto en la \textit{UX}\footnote{Experiencia de usuario}.
    \item Los usuarios deberían poder relacionarse con otros usuarios para poder gestionar la transferencia física de los teléfonos, para ello, habría que implementar un sistema que permitiese relacionarse con usuarios cercanos entre si.
    \item Permitir que los usuarios puedan poner nota a otros usuarios, de forma que permita que la comunidad se modere a si misma. Además, las ventajas de \textit{Blockchain} permiten limitar problemas como el \textit{review bombing}\footnote{Fenómeno de escribir reseñas negativas para dañar un producto}, al costar dinero el proceso de poner reseñas o las reseñas falsas al solamente dar permisos al usuario para escribir una reseña del producto que has comprado.
    \item Añadir en los contratos digitales el almacenamiento de objetos transacción. De esta forma, se puede asignar a estos objetos datos como comentarios sobre el estado de un producto comprado, fotos. Permitiendo crear un histórico del producto con más datos.
    \item Poder ver datos sobre los propietarios, reseñas recibidas, móviles a la venta, volumen de intercambios. Parte de esta información se podría extraer empleando la integración con \textit{Etherscan}, el objetivo sería ampliar esta información y facilitar su acceso al usuario.
    \item Desplegar el proyecto a la red principal, cualquier \textit{Dapp} aspira a ser migrada a la red principal que es la que garantiza que los datos sean permanentes, aunque por los motivos expuestos en 
    anteriormente en esta memoria esto tiene unos costes económicos muy grandes.
\end{itemize}