\capitulo{2}{Objetivos del proyecto}

\section{Objetivos teóricos}

\begin{enumerate}
  \item Estudiar el funcionamiento de redes \textit{blockchain} y su aplicación en entornos de compraventa.
  \item Crear un \textit{smart contract} capaz de satisfacer los objetivos técnicos del proyecto.
  \item Emplear un sistema que permita mantener la coherencia al resolver el problema de mantener los datos descentralizados gracias a \textit{blockchain} y la imposibilidad por costes de almacenar archivos grandes como imágenes.
  \item Crear una aplicación web capaz de aprovechar todas las características del \textit{smart contract}.
  \item Investigar y resolver la problemática de desplegar el \textit{smart contract} en una \textit{blockchain}  real.
  \item Aprender el funcionamiento de la librería de \textit{Javascript} \textit{React} para desarrollar el \textit{frontend}.
  \item Investigar y emplear herramientas para estilar de la forma más óptima posible la web.
  \item Emplear la metodología Scrum en la gestión del proyecto.
\end{enumerate}


\section{Objetivos técnicos}

\begin{itemize}
  \item Seguir el estándar \textit{ERC721} para el desarrollo del \textit{smart contract} que permitiría, si se desease en un futuro, su completa interoperabilidad con otros \textit{smart contracts}.
  \item El usuario debe ser capaz de registrar teléfonos móviles de forma única por dispositivo, es decir que no se permita que un dispositivo esté registrado varias veces.
  \item Se debe poder añadir una imagen por dispositivo, asegurando que el almacenamiento de esta siga descentralizado, manteniendo la coherencia con la información guardada en \textit{blockchain}.
  \item Los usuarios podrán seguir sus transacciones desde \textit{Etherscan}, una web que permite explorar el estado de la \textit{blockchain}.
  \item La aplicación debe implementar un \textit{marketplace}, capaz de poner teléfonos a la venta, editar su precio, retirarlos de la venta y poder comprarse por otros usuarios.
  \item Desplegar el \textit{smart contract} en una \textit{testnet} de \textit{Ethereum}.
  \item Desplegar la aplicación web que interactuará con el \textit{smart contract} desplegado en la \textit{testnet}.
  \item Para poder realizar transacciones la web debe emplear la \textit{wallet Metamask} que permita de forma segura al usuario firmar con su clave privada las transacciones.
  \item Realizar un diseño de la aplicación web \textit{responsive} que permita emplear la aplicación desde otros dispositivos como smartphones.
  \item Emplear el superset de \textit{Javascript}, \textit{TypeScript} desarrollado por \textit{Microsoft} para poder utilizar tipado, mejorando el código y la detección temprana de errores.
\end{itemize}

\section{Objetivos personales}
\begin{itemize}
  \item Continuar familiarizándome con el desarrollo web empleando librerías como \textit{React}.
  \item Investigar y comprender el funcionamiento de la tecnología \textit{blockchain}.
  \item Aprender a crear una \textit{dApp}\footnote{Aplicación descentralizada}.
  \item Intentar desarrollar la aplicación de la forma lo más descentralizada posible, manteniendo la coherencia de las ventajas de las redes \textit{blockchain}, de forma que la propiedad de los datos no pueda ser alterada por actores a los que no les pertenecen.
\end{itemize}

\section{Entregables}
\begin{itemize}
    \item Memoria y Anexos del proyecto.
    \item Proyecto desplegado en: https://second-hand-chain.vercel.app/
    \item Repositorio público en \textit{Github}: "https://github.com/axelrg/second-hand-chain" .
    \item Repositorio privado (Con acceso para el tribunal) en \textit{GitLab}:
        "https://gitlab.com/HP-SCDS/Observatorio/2022-2023/secondhandchain/ubu-secondhandchain" .
    \item Videos para ayudar a comprender el funcionamiento en la web desplegada.
    \item Tres USB con el proyecto completo.
    \item Una copia de la memoria en formato físico.
\end{itemize}