\capitulo{1}{Introducción}

En primer lugar me gustaría agradecer a los tutores de este proyecto su colaboración, por lo tanto, muchas gracias por su contribución a Jesús Maudes Raedo y Sandra Rodríguez Arribas por parte de la Universidad de Burgos. Además, por parte de HP SCDS a Julia Zuara Jiménez y a Daniel López Palomo por proponer este maravilloso proyecto y su contribución a lo largo de su desarrollo.

\noindent\rule{4cm}{0.5pt}

\textit{Second Hand Chain} es un proyecto que gestiona la compraventa de teléfonos móviles de segunda mano, implementando características que aporten valor a los distintos actores en el mercado a lo largo de la vida de un dispositivo.

Para ello se emplea la tecnología \textit{blockchain} mediante el uso de \textit{smart contracts} para gestionar de forma transparente las diferentes transacciones, permitiendo que esta información sea pública e inmutable por ningún tercero debido a las características de diseño de estas. Esta tecnología aporta a los usuarios la garantía de que esos datos no han sido alterados, resolviendo una de las grandes preocupaciones de los usuarios que realizan acciones de compraventa de objetos de gran valor en el mercado de segunda mano.

Además el proyecto implementa una aplicación web que permite interactuar con el \textit{smart contract}, haciendo accesible el proyecto a la mayoría de usuarios. Esta web actualmente está \href{https://second-hand-chain.vercel.app/}{desplegada}\footnote{https://second-hand-chain.vercel.app/} y permite interactuar con el \textit{smart contract} a cualquier usuario que acceda a ella.

Dicha web, permite a los usuarios el registro, la puesta a la venta y la compra de teléfonos móviles además de el cambio de precio o la retirada de la venta. Debido a las características de \textit{blockchain} permite a los usuarios disponer una gran cantidad de la información de gran valor debido precisamente a la inmutabilidad de esta información.

Actualmente el \textit{smart contract} se encuentra desplegado en una \textit{testnet de ethereum}\footnote{Red blockchain empleada por desarrolladores y aplicaciones en pruebas que permite no pagar por las transacciones, a diferencia de la red principal} llamada \textit{Sepolia} que se comporta exactamente igual que la red principal y permite que los usuarios y especialmente desarrolladores puedan probar el \textit{smart contract} sin el pago de \textit{gas fees}\footnote{Costes por realizar transacciones en Blockchain}, al poder obtener de forma gratuita la criptomoneda de la red y evitando el costoso despliegue en la red principal, que dependiendo del tamaño de los contratos puede llegar a los miles de euros.

\section{Estructura de la memoria}

\begin{itemize}
  \item \textbf{Objetivos del trabajo}: En esta parte se van a explicar cuales son los objetivos en este proyecto.
  \item \textbf{Conceptos teóricos}: En esta sección se va a explicar cual es el funcionamiento de las redes \textit{blockchain}, para ello se empezara explicando el caso de \textit{Bitcoin} siguiendo su evolución hasta \textit{Ethereum} y los \textit{smart contracts}. Se busca explicar por que estas redes son inmutables y por ello perfectas para el proyecto.
  \item \textbf{Técnicas y herramientas}: En este apartado se va a describir las herramientas utilizadas y el motivo de su elección, exponiendo sus ventajas y desventajas respecto a otras alternativas.
  \item \textbf{Aspectos relevantes del desarrollo}: En este capitulo se explicará las decisiones tomadas durante el diseño y el desarrollo del proyecto además de otros aspectos que son relevantes.
  \item \textbf{Trabajos relacionados}: En esta parte se buscará que proyectos hay similares y una comparación entre las características de cada uno.
  \item \textbf{Conclusiones y líneas futuras}: Finalmente se explicará que características nuevas puede añadir el proyecto. Además repasaremos todas las lecciones aprendidas durante el desarrollo del mismo.
\end{itemize}
